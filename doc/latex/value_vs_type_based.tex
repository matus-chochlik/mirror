\documentclass[compress,table,xcolor=table]{beamer}
\input{beamer_mirror.tex}
\begin{document}
% ------------------------------------------------------------------------------
\title{Value- vs. type-based reflection}
% ------------------------------------------------------------------------------
\section{reflection}
% ------------------------------------------------------------------------------
\begin{frame}[fragile]
  \frametitle{Reflection}
  \framesubtitle{Metaobjects}
  \begin{columns}
    \begin{column}{.05\textwidth}
    \rotatebox{90}{\Large value-based}
    \end{column}
    \begin{column}{.85\textwidth}
    \normalsize
    Reflection uses constant values of a single built-in type:
    \begin{lstlisting}[language=c++2x,basicstyle=\normalsize\ttfamily]
(*@\listinghls{info}@*) x = (*@\listinghl{reflect}@*)((*@\listinghlt{argument}@*));
    \end{lstlisting}
    \end{column}
  \end{columns}
  \vfill
  \hrule
  \vfill
  \begin{columns}
    \begin{column}{.05\textwidth}
    \rotatebox{90}{\Large type-based}
    \end{column}
    \begin{column}{.85\textwidth}
    \normalsize
    Reflection uses constant values of multiple types:
    \begin{lstlisting}[language=c++2x,basicstyle=\normalsize\ttfamily]
(*@\listinghls{auto}@*) x = (*@\listinghl{reflect}@*)((*@\listinghlt{argument}@*));
    \end{lstlisting}
    The actual (implementation-defined) type might be:
    \begin{lstlisting}[language=c++2x,basicstyle=\footnotesize\ttfamily]
template <(*@\listinghls{info}@*) X>
struct __metaobject {
  consteval operator (*@\listinghls{info}@*)() const {
    return X;
  }
};
    \end{lstlisting}
    \end{column}
  \end{columns}
\end{frame}
% ------------------------------------------------------------------------------
\begin{frame}[fragile]
  \frametitle{Reflection}
  \framesubtitle{APIs}
  \begin{columns}
    \begin{column}{.05\textwidth}
    \rotatebox{90}{\Large value-based}
    \end{column}
    \begin{column}{.85\textwidth}
    \normalsize
    Either purely \inlinecode{consteval} or template functions with
    non-type template parameters:
    \begin{lstlisting}[language=c++2x,basicstyle=\normalsize\ttfamily]
consteval auto foo((*@\listinghls{info}@*) mo);
    \end{lstlisting}
    \begin{lstlisting}[language=c++2x,basicstyle=\normalsize\ttfamily]
template <(*@\listinghls{info}@*) MO>
consteval auto bar();
    \end{lstlisting}
    \end{column}
  \end{columns}
  \hrule
  \vfill
  \begin{columns}
    \begin{column}{.05\textwidth}
    \rotatebox{90}{\Large type-based}
    \end{column}
    \begin{column}{.85\textwidth}
    \normalsize
    \normalsize
    \inlinecode{consteval} template functions taking metaobjects as function arguments:
    \begin{lstlisting}[language=c++2x,basicstyle=\normalsize\ttfamily]
consteval auto foo((*@\listinghl{metaobject}@*) (*@\listinghls{auto}@*) mo);
consteval auto bar((*@\listinghl{metaobject}@*) (*@\listinghls{auto}@*) mo);
    \end{lstlisting}
    \begin{lstlisting}[language=c++2x,basicstyle=\normalsize\ttfamily]
template <typename T>
concept (*@\listinghl{metaobject}@*) = (*@\listinghlt{unspecified}@*);
    \end{lstlisting}
    \end{column}
  \end{columns}
\end{frame}
% ------------------------------------------------------------------------------
\begin{frame}[fragile]
  \frametitle{Reflection}
  \framesubtitle{Implementation}
  \begin{columns}
    \begin{column}{.05\textwidth}
    \rotatebox{90}{\Large value-based}
    \end{column}
    \begin{column}{.85\textwidth}
    \normalsize
    Very fast to compile:
    \begin{lstlisting}[language=c++2x,basicstyle=\normalsize\ttfamily]
consteval auto foo((*@\listinghls{info}@*) mo);
    \end{lstlisting}
    Somewhat slower to compile:
    \begin{lstlisting}[language=c++2x,basicstyle=\normalsize\ttfamily]
template <(*@\listinghls{info}@*) MO>
consteval auto bar();
    \end{lstlisting}
    \end{column}
  \end{columns}
  \hrule
  \vfill
  \begin{columns}
    \begin{column}{.05\textwidth}
    \rotatebox{90}{\Large type-based}
    \end{column}
    \begin{column}{.85\textwidth}
    \normalsize
    Very fast to compile, but requires \inlinecode{consteval} conversion
    from \inlinecode{metaobject} to \inlinecode{info}:
    \begin{lstlisting}[language=c++2x,basicstyle=\normalsize\ttfamily]
consteval auto foo((*@\listinghls{info}@*) mo);
    \end{lstlisting}
    Slower to compile:
    \begin{lstlisting}[language=c++2x,basicstyle=\small\ttfamily]
template <(*@\listinghls{info}@*) (*@\listinghlt{MO}@*)>
consteval auto bar((*@\listinghl{__metaobject}@*)<(*@\listinghlt{MO}@*)> mo);
    \end{lstlisting}
    \end{column}
  \end{columns}
\end{frame}
% ------------------------------------------------------------------------------
\begin{frame}[fragile]
  \frametitle{Reflection}
  \framesubtitle{Usage}
  \begin{columns}
    \begin{column}{.05\textwidth}
    \rotatebox{90}{\Large value-based}
    \end{column}
    \begin{column}{.85\textwidth}
    \normalsize
    Dual syntax:
    \begin{lstlisting}[language=c++2x,basicstyle=\normalsize\ttfamily]
use(foo((*@\listinghl{reflect}@*)((*@\listinghlt{argument}@*))));
    \end{lstlisting}
    or
    \begin{lstlisting}[language=c++2x,basicstyle=\normalsize\ttfamily]
use(bar<(*@\listinghl{reflect}@*)((*@\listinghlt{argument}@*))>());
    \end{lstlisting}
    When to use which?
    \end{column}
  \end{columns}
  \vfill
  \hrule
  \vfill
  \begin{columns}
    \begin{column}{.05\textwidth}
    \rotatebox{90}{\Large type-based}
    \end{column}
    \begin{column}{.85\textwidth}
    \normalsize
    Uniform syntax:
    \begin{lstlisting}[language=c++2x,basicstyle=\normalsize\ttfamily]
use(foo((*@\listinghl{reflect}@*)((*@\listinghlt{argument}@*))));
    \end{lstlisting}
    and
    \begin{lstlisting}[language=c++2x,basicstyle=\normalsize\ttfamily]
use(bar((*@\listinghl{reflect}@*)((*@\listinghlt{argument}@*))));
    \end{lstlisting}
    \end{column}
  \end{columns}
\end{frame}
% ------------------------------------------------------------------------------
\begin{frame}[fragile]
  \frametitle{Reflection}
  \framesubtitle{Containers}
  \begin{columns}
    \begin{column}{.05\textwidth}
    \rotatebox{90}{\Large value-based}
    \end{column}
    \begin{column}{.85\textwidth}
    \normalsize
    Vectors, etc. must be fixed to work in \inlinecode{consteval}
    \begin{lstlisting}[language=c++2x,basicstyle=\normalsize\ttfamily]
  vector<(*@\listinghls{info}@*)> x = members_of(...);
  vector<(*@\listinghls{info}@*)> y = bases_of(...);
    \end{lstlisting}
    Can be used with some STL algorithms, unless splicing is involved
    \end{column}
  \end{columns}
  \vfill
  \hrule
  \vfill
  \begin{columns}
    \begin{column}{.05\textwidth}
    \rotatebox{90}{\Large type-based}
    \end{column}
    \begin{column}{.85\textwidth}
    \normalsize
      Containers (sequences) are metaobjects themselves
    \begin{lstlisting}[language=c++2x,basicstyle=\normalsize\ttfamily]
  (*@\listinghls{auto}@*) x = get_data_members(...);
  (*@\listinghls{auto}@*) y = get_base_classes(...);
    \end{lstlisting}
    Have their own implementation of reflection-related algorithms, splicing
      is no problem
    \end{column}
  \end{columns}
\end{frame}
% ------------------------------------------------------------------------------
\begin{frame}
  \frametitle{Reflection}
  \framesubtitle{Pros}
  \begin{columns}
    \begin{column}{.05\textwidth}
    \rotatebox{90}{\Large value-based}
    \end{column}
    \begin{column}{.85\textwidth}
    \large
    \begin{itemize}
      \item Faster to compile
      \item Uses less resources to compile
    \end{itemize}
    \end{column}
  \end{columns}
  \vfill
  \hrule
  \vfill
  \begin{columns}
    \begin{column}{.05\textwidth}
    \rotatebox{90}{\Large type-based}
    \end{column}
    \begin{column}{.85\textwidth}
    \large
    \begin{itemize}
      \item Consistent and unified API
      \item More friendly to generic programming
      \item Plays better with ADL
      \item Better usability
      \item Easier to teach
    \end{itemize}
    \end{column}
  \end{columns}
\end{frame}
% ------------------------------------------------------------------------------
\begin{frame}
  \frametitle{Reflection}
  \framesubtitle{Cons}
  \begin{columns}
    \begin{column}{.05\textwidth}
    \rotatebox{90}{\Large value-based}
    \end{column}
    \begin{column}{.85\textwidth}
    \large
    \begin{itemize}
      \item Inconsistent API
      \item The \inlinecode{foo(...)} vs. \inlinecode{bar<...>()} syntax
        makes it less generic
      \item Rules when to use which, are sort of complicated and may look arbitrary
      \item More complicated to teach
      \item Issues with ADL on NTTPs
    \end{itemize}
    \end{column}
  \end{columns}
  \vfill
  \hrule
  \vfill
  \begin{columns}
    \begin{column}{.05\textwidth}
    \rotatebox{90}{\Large type-based}
    \end{column}
    \begin{column}{.85\textwidth}
    \large
    \begin{itemize}
      \item Slower to compile
      \item Uses more resources to compile
    \end{itemize}
    \end{column}
  \end{columns}
\end{frame}
% ------------------------------------------------------------------------------
\section{usability}
% ------------------------------------------------------------------------------
\begin{frame}[fragile]
  \frametitle{Usability issues}
  \framesubtitle{the dual value-based API}
  \begin{lstlisting}[language=c++2x,basicstyle=\small\ttfamily]
// span<meta::info>, vector<meta::info>
auto (*@\listinghls{mem}@*) = members_of(^T);
auto (*@\listinghlt{func}@*) = // some callable, example later
  \end{lstlisting}
  The following is possible only for a subset of possible reflection
  operations:
  \begin{lstlisting}[language=c++2x,basicstyle=\small\ttfamily]
std::count_if(*@\footnote{
  or any other of countless possible algorithms}@*)((*@\listinghls{mem}@*).begin(), (*@\listinghls{mem}@*).end(), func);
  \end{lstlisting}
  Specifically the \inlinecode{func} cannot use splicing, because
  \inlinecode{count_if} will call it as:
  \begin{lstlisting}[language=c++2x,basicstyle=\small\ttfamily]
(*@\listinghlt{func}@*)((*@\listinghl{element}@*));
  \end{lstlisting}
  and not as:
  \begin{lstlisting}[language=c++2x,basicstyle=\small\ttfamily]
(*@\listinghlt{func}@*)<(*@\listinghl{element}@*)>();
  \end{lstlisting}
\end{frame}
% ------------------------------------------------------------------------------
\begin{frame}[fragile]
  \frametitle{Usability issues}
  \framesubtitle{writing generic algorithms}
  Users\footnote{and library authors} will want to write their own reusable
  algorithms, that take other functions\footnote{predicates, transforms, etc.}
  as their arguments:
  \begin{lstlisting}[language=c++2x,basicstyle=\small\ttfamily]
consteval void my_reusable_algo(
  span<meta::info> (*@\listinghls{s}@*),
  function<bool(meta::info)> (*@\listinghlt{predicate}@*),
  function<void(meta::info)> (*@\listinghlt{function}@*)) {
  for(auto (*@\listinghl{e}@*) : (*@\listinghls{s}@*)) {
    if((*@\listinghlt{predicate}@*)((*@\listinghl{e}@*)) && something_else((*@\listinghl{e}@*))) {
      (*@\listinghlt{function}@*)((*@\listinghl{e}@*));
    }
  }
}
  \end{lstlisting}
  \inlinecode{predicate}, \inlinecode{something_else} and \inlinecode{function}
  {\em \larger cannot} do~splicing\ldots
\end{frame}
% ------------------------------------------------------------------------------
\begin{frame}[fragile]
  \frametitle{Usability issues}
  \framesubtitle{supporting splicing}
  \ldots to support splicing we'd have to:
  \begin{lstlisting}[language=c++2x,basicstyle=\small\ttfamily]
template <auto (*@\listinghls{s}@*)>
consteval void my_reusable_algo(
  auto (*@\listinghlt{predicate}@*),
  auto (*@\listinghlt{function}@*)) {
  template for(auto (*@\listinghl{e}@*) : (*@\listinghls{s}@*)) {
    if((*@\listinghlt{predicate}@*)<(*@\listinghl{e}@*)>() && something_else<(*@\listinghl{e}@*)>()) {
      (*@\listinghlt{function}@*)<(*@\listinghl{e}@*)>();
    }
  }
}
  \end{lstlisting}
  making everything a template. But then this becomes slower to compile,
  partially defeating one of the main points of this API.
\end{frame}
% ------------------------------------------------------------------------------
\begin{frame}
  \frametitle{BTW, why so much focus on splicing?}
  \framesubtitle{some anecdotes\ldots}
  \begin{itemize}
  \item Out of these use-cases\footnote{all implemented here:
    \url{https://github.com/matus-chochlik/mirror}}\footnote{
      and there is a whole other presentation about the details}
    \begin{itemize}
      \smaller
      \item enum / string conversion,
      \item serialization and deserialization,
      \item parsing of command line arguments into a config structure,
      \item RPC stubs and skeletons,
      \item generic wrapper for a REST API,
      \item automated registering with a scripting engine,
      \item generating UML diagrams from code,
      \item fetching and converting data from an SQL database,
      \item generating SQL queries from the names in an \say{interface} class,
      \item implementation of the factory pattern.
    \end{itemize}
  \item All but one\footnote{UML generation} required splicing
  \item Various forms of splicing are {\em \larger very} common in use-cases
  \end{itemize}
\end{frame}
% ------------------------------------------------------------------------------
\section{setup}
% ------------------------------------------------------------------------------
\begin{frame}[fragile]
  \frametitle{What are we trying to do?}
  Determine what is the actual overhead of this:
  \begin{lstlisting}[language=c++2x,basicstyle=\footnotesize\ttfamily]
template <(*@\listinghls{info}@*) X>
struct (*@\listinghlt{__metaobject}@*) {
  consteval operator (*@\listinghls{info}@*)() const { return X; }
};
concept (*@\listinghlt{metaobject}@*) = (*@\listinghls{unspecified}@*);

consteval auto foo(info mo);
consteval auto bar((*@\listinghlt{metaobject}@*) auto mo);
  \end{lstlisting}
  compared to this:
  \begin{lstlisting}[language=c++2x,basicstyle=\footnotesize\ttfamily]
consteval auto foo((*@\listinghls{info}@*) mo);

template <(*@\listinghls{info}@*) MO>
consteval auto bar();
  \end{lstlisting}
  in a real-life scenario.
\end{frame}
% ------------------------------------------------------------------------------
\begin{frame}
  \frametitle{The cost of reflection in a \say{large-ish} project?}
  \LARGE
  \begin{itemize}
    \item Let's try \inlinecode{clang}
    \begin{itemize}
      \Large
      \item Estimate the number of \say{things} to reflect
      \item Measure the overall compilation time
      \item Measure the contribution of reflection
      \item Compare purely value-based and typed metaobjects
    \end{itemize}
  \end{itemize}
\end{frame}
% ------------------------------------------------------------------------------
\begin{frame}[fragile]
  \frametitle{How to materialize 100'000s of metaobjects?}
  Use a shell script\ldots
  \begin{lstlisting}[language=bash,basicstyle=\scriptsize\ttfamily]
L=100  # number of repeats
S=1000 # sampling step size
for l in $(seq 1 ${L})
do
  (*@\listinghl{N}@*)=$((l * S))
  # factorize N into three integers
  (*@\listinghl{D}@*)=...; (*@\listinghl{E}@*)=...; (*@\listinghl{F}@*)=...
  \end{lstlisting}
  \ldots to generate a C++ source file\ldots
  \begin{lstlisting}[language=c++2x,basicstyle=\scriptsize\ttfamily]
  int main() {
    return bool((*@\listinghlt{qux}@*)(make_index_sequence<(*@\listinghl{\$\{D\}}@*)>{})) ? 0 : 1;
  }
  \end{lstlisting}
  \ldots, compile and measure:
  \begin{lstlisting}[language=bash,basicstyle=\scriptsize\ttfamily]
  time $(CXX) $(CXXFLAGS) -o /dev/null $<
done
  \end{lstlisting}
\end{frame}
% ------------------------------------------------------------------------------
\begin{frame}[fragile]
  \frametitle{The boilerplate}
  \framesubtitle{level 1}
  \begin{lstlisting}[language=c++2x,basicstyle=\normalsize\ttfamily]
template <size_t ... K>
consteval auto (*@\listinghlt{qux}@*)(index_sequence<(*@\listinghls{K}@*)...>) {
	return ( ... + (*@\listinghlt{baz}@*)(
		integral_constant<size_t, (*@\listinghls{K}@*)>{},
		make_index_sequence<(*@\listinghl{\$\{E\}}@*)>{}));
}
  \end{lstlisting}
\end{frame}
% ------------------------------------------------------------------------------
\begin{frame}[fragile]
  \frametitle{The boilerplate}
  \framesubtitle{level 2}
  \begin{lstlisting}[language=c++2x,basicstyle=\normalsize\ttfamily]
template <size_t K, size_t ... J>
consteval auto (*@\listinghlt{baz}@*)(
	integral_constant<size_t, (*@\listinghls{K}@*)>,
	index_sequence<(*@\listinghls{J}@*)...>) {
	return ( ... + (*@\listinghlt{bar}@*)(
		integral_constant<size_t, (*@\listinghls{K}@*)>{},
		integral_constant<size_t, (*@\listinghls{J}@*)>{},
		make_index_sequence<(*@\listinghl{\$\{F\}}@*)>{}));
}
  \end{lstlisting}
\end{frame}
% ------------------------------------------------------------------------------
\begin{frame}[fragile]
  \frametitle{The boilerplate}
  \framesubtitle{level 3}
  \begin{lstlisting}[language=c++2x,basicstyle=\normalsize\ttfamily]
template <size_t K, size_t J, size_t ... I>
consteval auto (*@\listinghlt{bar}@*)(
	integral_constant<size_t, (*@\listinghls{K}@*)>,
	integral_constant<size_t, (*@\listinghls{J}@*)>,
	index_sequence<(*@\listinghls{I}@*)...>) {
    // Simulate the metaobject "id" as:
    // (*@\listinghls{MOID}@*) = 
    //   (*@\listinghls{K}@*) * $(((*@\listinghl{N}@*) / (*@\listinghl{D}@*))) +
    //   (*@\listinghls{J}@*) * $(((*@\listinghl{N}@*) / ((*@\listinghl{D}@*) * (*@\listinghl{E}@*)))) +
    //   (*@\listinghls{I}@*);
    return /* Do something with (*@\listinghls{MOID}@*)... */
}
  \end{lstlisting}
\end{frame}
% ------------------------------------------------------------------------------
\begin{frame}[fragile]
  \frametitle{The baseline}
  Just sum the {\em MOID} values at compile-time
  \begin{lstlisting}[language=c++2x,basicstyle=\normalsize\ttfamily]
template <size_t K, size_t J, size_t ... I>
consteval auto (*@\listinghlt{bar}@*)(
	integral_constant<size_t, (*@\listinghls{K}@*)>,
	integral_constant<size_t, (*@\listinghls{J}@*)>,
	index_sequence<(*@\listinghls{I}@*)...>) {
    return (... + 
      (*@\listinghls{K}@*) * $(((*@\listinghl{N}@*) / (*@\listinghl{D}@*))) +
      (*@\listinghls{J}@*) * $(((*@\listinghl{N}@*) / ((*@\listinghl{D}@*) * (*@\listinghl{E}@*)))) +
      (*@\listinghls{I}@*));
}
  \end{lstlisting}
  Measure how long does this take to compile and subtract from \say{real}
  measurements.
\end{frame}
% ------------------------------------------------------------------------------
\begin{frame}[fragile]
  \frametitle{Type-based metaobject \& template function}
  \begin{lstlisting}[language=c++2x,basicstyle=\normalsize\ttfamily]
template <size_t M>
struct (*@\listinghl{wrapper}@*) {
	consteval operator size_t() const {
		return M;
	}
};

template <size_t M>
consteval size_t (*@\listinghlt{foo}@*)((*@\listinghl{wrapper}@*)<M> w) {
	return w;
}
  \end{lstlisting}
  \vfill
  \begin{lstlisting}[language=c++2x,basicstyle=\normalsize\ttfamily]
return ( ... + (*@\listinghlt{foo}@*)((*@\listinghl{wrapper}@*)<(*@\listinghls{MOID}@*)>{}));
  \end{lstlisting}
\end{frame}
% ------------------------------------------------------------------------------
\begin{frame}[fragile]
  \frametitle{Type-based metaobject \& consteval function}
  \begin{lstlisting}[language=c++2x,basicstyle=\normalsize\ttfamily]
template <size_t M>
struct (*@\listinghl{wrapper}@*) {
	consteval operator size_t() const {
		return M;
	}
};

consteval size_t (*@\listinghlt{foo}@*)(size_t m) {
	return m;
}
  \end{lstlisting}
  \vfill
  \begin{lstlisting}[language=c++2x,basicstyle=\normalsize\ttfamily]
return ( ... + (*@\listinghlt{foo}@*)((*@\listinghl{wrapper}@*)<(*@\listinghls{MOID}@*)>{}));
  \end{lstlisting}
\end{frame}
% ------------------------------------------------------------------------------
\begin{frame}[fragile]
  \frametitle{Value-based metaobject \& template function}
  \begin{lstlisting}[language=c++2x,basicstyle=\normalsize\ttfamily]
template <size_t M>
consteval size_t (*@\listinghlt{foo}@*)() {
	return M;
}
  \end{lstlisting}
  \vfill
  \begin{lstlisting}[language=c++2x,basicstyle=\normalsize\ttfamily]
return ( ... + (*@\listinghlt{foo}@*)<(*@\listinghls{MOID}@*)>());
  \end{lstlisting}
\end{frame}
% ------------------------------------------------------------------------------
\begin{frame}[fragile]
  \frametitle{Value-based metaobject \& consteval function}
  \begin{lstlisting}[language=c++2x,basicstyle=\normalsize\ttfamily]
consteval size_t (*@\listinghlt{foo}@*)(size_t i) {
	return i;
}
  \end{lstlisting}
  \vfill
  \begin{lstlisting}[language=c++2x,basicstyle=\normalsize\ttfamily]
return ( ... + (*@\listinghlt{foo}@*)((*@\listinghls{MOID}@*)));
  \end{lstlisting}
\end{frame}
% ------------------------------------------------------------------------------
\begin{frame}
  \frametitle{Test hardware}
  \larger
  \begin{itemize}
  \item Old desktop\footnote{2010}:
    \begin{itemize}
    \smaller
    \item i5-2400U @ 3.10GHz (4 cores)
    \item 24GB RAM
    \end{itemize}
  \item Corporate dev laptop\footnote{2021}:
    \begin{itemize}
    \smaller
    \item i7-1185G7 @ 3.0GHz (8 cores)
    \item 32GB RAM
    \end{itemize}
  \item Mid-range gaming laptop\footnote{2019}:
    \begin{itemize}
    \smaller
    \item AMD Ryzen7 4800HS (16 cores)
    \item 16GB RAM
    \end{itemize}
  \item RPi 4B\footnote{timeless}
    \begin{itemize}
    \smaller
    \item ARM v7l
    \item 4GB RAM
    \end{itemize}
  \end{itemize}
\end{frame}
% ------------------------------------------------------------------------------
\section{measurements}
% ------------------------------------------------------------------------------
\begin{frame}
  \frametitle{i5-2400 -- compile time increase per N metaobjects}
  \fitfig{ct-per-n-i5-2400}
\end{frame}
% ------------------------------------------------------------------------------
\begin{frame}
  \frametitle{i5-2400 -- compile time increase per 1 metaobject}
  \fitfig{ct-per-1-i5-2400}
\end{frame}
% ------------------------------------------------------------------------------
\begin{frame}
  \frametitle{i5-2400 -- How much faster is VB vs. TB}
  \fitfig{ct-ratios-i5-2400}
\end{frame}
% ------------------------------------------------------------------------------
\begin{frame}
  \frametitle{i7-1185 -- compile time increase per N metaobjects}
  \fitfig{ct-per-n-i7-1185}
\end{frame}
% ------------------------------------------------------------------------------
\begin{frame}
  \frametitle{i7-1185 -- compile time increase per 1 metaobject}
  \fitfig{ct-per-1-i7-1185}
\end{frame}
% ------------------------------------------------------------------------------
\begin{frame}
  \frametitle{i7-1185 -- How much faster is VB vs. TB}
  \fitfig{ct-ratios-i7-1185}
\end{frame}
% ------------------------------------------------------------------------------
\begin{frame}
  \frametitle{Ryzen7-4800HS-- compile time increase per N metaobjects}
  \fitfig{ct-per-n-R7-4800}
\end{frame}
% ------------------------------------------------------------------------------
\begin{frame}
  \frametitle{Ryzen7-4800HS -- compile time increase per 1 metaobject}
  \fitfig{ct-per-1-R7-4800}
\end{frame}
% ------------------------------------------------------------------------------
\begin{frame}
  \frametitle{Ryzen7-4800HS -- How much faster is VB vs. TB}
  \fitfig{ct-ratios-R7-4800}
\end{frame}
% ------------------------------------------------------------------------------
\begin{frame}
  \frametitle{ARMv7-- compile time increase per N metaobjects}
  \fitfig{ct-per-n-ARMv7l}
\end{frame}
% ------------------------------------------------------------------------------
\begin{frame}
  \frametitle{ARMv7-- compile time increase per 1 metaobject}
  \fitfig{ct-per-1-ARMv7l}
\end{frame}
% ------------------------------------------------------------------------------
\begin{frame}
  \frametitle{ARMv7 -- How much faster is VB vs. TB}
  \fitfig{ct-ratios-ARMv7l}
\end{frame}
% ------------------------------------------------------------------------------
% TODO describe the "copy/paste" setup
% ------------------------------------------------------------------------------
\begin{frame}
  \frametitle{Ryzen7-4800HS-- compile time increase per N metaobjects}
  \fitfig{ct-per-n-R7-4800-cp}
\end{frame}
% ------------------------------------------------------------------------------
\begin{frame}
  \frametitle{Ryzen7-4800HS -- compile time increase per 1 metaobject}
  \fitfig{ct-per-1-R7-4800-cp}
\end{frame}
% ------------------------------------------------------------------------------
\begin{frame}
  \frametitle{Ryzen7-4800HS -- How much faster is VB vs. TB}
  \fitfig{ct-ratios-R7-4800-cp}
\end{frame}
% ------------------------------------------------------------------------------
\begin{frame}
  \frametitle{What about executable sizes?}
  \Large
  \begin{itemize}
    \item This is boring\ldots
    \item When the reflection-related functions are \inlinecode{consteval},
      the executable size stays the same regardless of the representation
      of metaobjects or their count
    \item The test source code shown above always compiles into an~executable
      roughly 16kB  in size

  \end{itemize}
\end{frame}
% ------------------------------------------------------------------------------
\section{clang}
% ------------------------------------------------------------------------------
\begin{frame}[fragile]
  \frametitle{Estimating number of declarations in \inlinecode{clang}}
  Let's try {\em \larger documented} declarations
  \vfill
  Edit \inlinecode{doxygen-cfg.in}:
  \begin{lstlisting}[language=bash,basicstyle=\scriptsize\ttfamily]
- GENERATE_XML   = (*@\listinghls{NO}@*)
+ GENERATE_XML   = (*@\listinghlt{YES}@*)
  \end{lstlisting}
  Configure:
  \begin{lstlisting}[language=bash,basicstyle=\scriptsize\ttfamily]
cmake \
  -D(*@\listinghl{LLVM_ENABLE_DOXYGEN}@*)=(*@\listinghl{On}@*) \
  ...
  \end{lstlisting}
  Generate Doxygen docs:
  \begin{lstlisting}[language=bash,basicstyle=\scriptsize\ttfamily]
ninja doxygen-clang
  \end{lstlisting}
  Merge into a single XML file \inlinecode{clang.xml}: 
  \begin{lstlisting}[language=bash,basicstyle=\scriptsize\ttfamily]
xsltproc combine.xslt index.xml > (*@\listinghl{clang.xml}@*)
  \end{lstlisting}
\end{frame}
% ------------------------------------------------------------------------------
\begin{frame}[fragile]
  \frametitle{Counting documented declarations in \inlinecode{clang}}
  Create \inlinecode{count.xslt}:
  \begin{lstlisting}[language=xslt,basicstyle=\scriptsize\ttfamily]
<?xml version="1.0" encoding="utf8"?>
<xsl:stylesheet version = '1.0'
 xmlns:xsl='http://www.w3.org/1999/XSL/Transform'>
  <xsl:template match="/">
    <xsl:value-of select="(*@\listinghlt{count}@*)(
      descendant::(*@\listinghl{compounddef}@*)(*@\footnote{
        \inlinecode{struct}s, \inlinecode{class}es, \inlinecode{enum}s, \ldots}@*)|
      descendant::(*@\listinghl{member}@*)(*@\footnote{
        data members, member functions, enumerators, \ldots}@*)|
      descendant::(*@\listinghl{value}@*)(*@\footnote{
        enumerator values, default arguments, \ldots}@*)|
      descendant::(*@\listinghl{para}@*)(*@\footnote{
        function/constructor/operator parameters, \ldots}@*)|
      descendant::(*@\listinghl{param}@*)(*@\footnote{
        template parameters, \ldots}@*))
    "/>
  </xsl:template>
</xsl:stylesheet>
  \end{lstlisting}
\end{frame}
% ------------------------------------------------------------------------------
\begin{frame}[fragile]
  \frametitle{Counting documented declarations in \inlinecode{clang}}
  \begin{columns}
    \begin{column}{.35\textwidth}
      \Large
      Calculate!
      \begin{lstlisting}[language=bash,basicstyle=\small\ttfamily]
xsltproc \
    (*@\listinghlt{count.xslt}@*) \
    (*@\listinghl{clang.xml}@*)
      \end{lstlisting}
      The result:
      \begin{verbatim}
379091
      \end{verbatim}
    \end{column}
    \begin{column}{.55\textwidth}
      \large
      \begin{itemize}
        \item That's for version 15.0.0
        \item Around FEB-05-2022
        \item Round that up to 400'000, 500'000 or even 1'000'000
        \item Let's assume we want to reflect every single declaration
      \end{itemize}
    \end{column}
  \end{columns}
\end{frame}
% ------------------------------------------------------------------------------
\begin{frame}[fragile]
  \frametitle{Clean build of \inlinecode{clang}}
  Edit \inlinecode{toolchain.cmake}:
  \begin{lstlisting}[language=bash,basicstyle=\scriptsize\ttfamily]
set(LLVM_USE_LINKER (*@\listinghl{lld}@*))
set(CMAKE_EXE_LINKER_FLAGS -fuse-ld=${LLVM_USE_LINKER})
set(CMAKE_SHARED_LINKER_FLAGS -fuse-ld=${LLVM_USE_LINKER})
  \end{lstlisting}
  Configure:
  \begin{lstlisting}[language=bash,basicstyle=\footnotesize\ttfamily]
cmake \
  -DLLVM_ENABLE_PROJECTS="clang;clang-tools-extra" \
  -DLLVM_ENABLE_RUNTIMES="libcxx;libcxxabi" \
  -DLLVM_TOOLCHAIN_FILE="toolchain.cmake" \
  ...
  \end{lstlisting}
  Build and measure elapsed time:
  \begin{lstlisting}[language=bash,basicstyle=\footnotesize\ttfamily]
time ninja install install-cxx install-cxxabi
  \end{lstlisting}
\end{frame}
% ------------------------------------------------------------------------------
\begin{frame}
  \frametitle{Clean build of \inlinecode{clang}}
  \Large
  Results:
  \normalsize
  \begin{center}
  {\rowcolors{2}{mirrorpage}{darkgray}
    \begin{tabular}{|p{3cm}|p{2cm}|p{2cm}|p{2cm}|}
    \hline
      CPU: & i5-2400 & i7-1185 & Ryzen 7 \\
    \hline
      real & 122m25,943s &  66m59,909s &  34m45,899s \\
      user & 433m50,123s & 510m55,382s & 525m16,660s \\
      sys  &  11m22,881s &  12m52,287s &  17m5,738s  \\
    \hline
    \end{tabular}
  }
  \end{center}
  \vfill
  \Large
  Added, rounded and converted to seconds:
  \normalsize
  \begin{center}
  {\rowcolors{2}{mirrorpage}{darkgray}
    \begin{tabular}{|p{3.5cm}|p{1.5cm}|p{1.5cm}|p{1.5cm}|}
    \hline
      CPU: & i5-2400 & i7-1185 & Ryzen 7 \\
    \hline
      real-time           &  7346s &  4020s &  2086s \\
      cpu-time (user+sys) & 27313s & 31427s & 32543s \\
    \hline
    \end{tabular}
  }
  \end{center}
\end{frame}
% ------------------------------------------------------------------------------
\begin{frame}
  \frametitle{Compared to build-time with 400'000 metaobjects}
  \large
  Compile-time of a typical \inlinecode{clang} build vs.\\
  compile-time spent on materializing 400'000 metaobjects:
  \normalsize
  \begin{center}
  {\rowcolors{2}{mirrorpage}{darkgray}
    \begin{tabular}{|p{4.5cm}|p{1.5cm}|p{1.5cm}|p{1.5cm}|}
    \hline
      CPU:    & i5-2400 & i7-1185 & Ryzen 7 \\
    \hline
      clang:                   & 27313s  & 31427s  & 32543s  \\
    \hline
      400k MO (TB-template):   & 115.9s  &  48.8s  &  62.4s  \\
                               & 0.42\%  & 0.16\%  &  0.19\% \\
    \hline
      400k MO (TB-consteval):  & 111.3s  &  53.4s  &  62.7   \\
                               & 0.41\%  & 0.16\%  &  0.19\% \\
    \hline
      400k MO (VB-template):   &  36.3s  &  16.5s  &  19.0s  \\
                               & 0.13\%  & 0.05\%  &  0.06\% \\
    \hline
      400k MO (VB-consteval):  &  50.3s  &  27.4s  &  29.6s  \\
                               & 0.18\%  & 0.09\%  &  0.09\% \\
    \hline
    \end{tabular}
  }
  \end{center}
\end{frame}
% ------------------------------------------------------------------------------
\section{conclusion}
% ------------------------------------------------------------------------------
\begin{frame}
  \frametitle{Conclusions}
  \Large
  \begin{itemize}
    \item The typical compile-time overhead of materializing a metaobject
      is on the order of tens or hundreds of microseconds
    \item The type-based metaobject representation is between 2x and 4x
      slower to compile compared to the purely value-based representation
  \end{itemize}
\end{frame}
% ------------------------------------------------------------------------------
\begin{frame}
  \frametitle{Conclusions}
  \framesubtitle{(cont.)}
  \Large
  \begin{itemize}
    \item Most typical reflection use-cases don't require reflecting every
      declaration in a project
    \item Even if reflecting almost everything, the~overhead compared
      to~total build time is a~fraction of~a~percent even in~the~worst case
    \item For projects similar in complexity to \inlinecode{clang},
      this results in 1-2 minutes added to several hours of compilation-time
  \end{itemize}
\end{frame}
% ------------------------------------------------------------------------------
\begin{frame}
  \frametitle{Conclusions}
  \framesubtitle{(cont.)}
  \Large
  \begin{itemize}
    \item Some of the compile-time advantage of value-based API disappears,
      when splicing is involved
    \item In the value-based API splicing requires passing metaobject
      as non-type template arguments
    \item Splicing is quite common in various use-cases
  \end{itemize}
\end{frame}
% ------------------------------------------------------------------------------
\begin{frame}
  \frametitle{The {\larger big} question}
  \Large
  \centering
  Is the improvement in compile-time worth the decrease in usability
  of the value-based reflection API?
\end{frame}
% ------------------------------------------------------------------------------
\end{document}

